\documentclass{article}
\usepackage{amsmath,amssymb,amsbsy}
\usepackage{graphicx}
\usepackage[letterpaper,lmargin=2.5cm,rmargin=2.5cm,tmargin=2.5cm,bmargin=2.5cm]{geometry}
\usepackage{color}
\usepackage{listings}
\usepackage{listings-python-options}
\usepackage{subfig}

% some useful definitions:

\newcommand{\tr}{\intercal} % transpose
% we use boldface math characters often so let's create some shortcuts:
\newcommand{\X}{\mathbf{X}}
\newcommand{\x}{\mathbf{x}}
\newcommand{\y}{\mathbf{y}}
\newcommand{\w}{\mathbf{w}}
% some useful operators
\newcommand{\argmax}[1]{\underset{#1}{\operatorname{argmax}}}
\newcommand{\argmin}[1]{\underset{#1}{\operatorname{argmin}}}




\begin{document}

\title{ Playing Card Image Recognition\\[1ex] \footnotesize\mdseries CS 545, Fall 2016 Project Proposal }
\author{Gregory Poisson, Bradley Winters, Joshua O'Dell}
\maketitle

\noindent\hrulefill
\vspace{-5mm} %to remove some whitespace before "Contents"
% \tableofcontents
% \noindent\hrulefill

\section{Project Type:  Existing algorithm/new problem}
We will train a neural network to recognize playing cards within an image.  The goal being to eventually utilize this knowledge to create a system which a human could use as an opponent in a variety of games involving playing cards, for practice or potentially for competition.

JO: some ideas, the goal would be to create an AI assitant to help you in learning poker?

\section{Project Introduction}
The first step of the project will involve significant data collection. We need a large number of images of playing cards to be used as training and as testing data. Since we shall need to train over so many of them, these images cannot be of very high resolution. Our rough procedure for data collection is as follows (we can format this to look a little cleaner, I'm just listing things for now): 1) Photograph each playing card from ten different angles and zoom levels. Ensure each image is labeled according to the card in the image, and that for each card, at least one labeled corner of the card is visible. This should produce 520 labeled images. 2) Using some simple scripting and the Linux convert tool, effects and transformations will be applied to copies of each of the images in order to increase the size of the dataset to roughly 10,000 labeled images. This will be the data we use for training and for testing. 3) A convolutional neural network \cite{conv} will take as input the pixel values for each image, and training will iterate until a model converges which can correctly classify a card in a new image. We will experiment with regularization parameters to prevent the network from learning overly specific features, as well as with the size of the hidden layers. The network will have a number of inputs equal to the number of pixels in an image (currently planning something around the 150x150 range), and will have 52 outputs, one for each possible card.

It will be interesting to see if the network is able to identify multiple cards in the same image, and to see if it can make reasonable guesses about cards which may be partially obscured (ie, only the number is visible, but not the suit, etc).

To implement the neural network, we can use a modern machine learning framework such as Theano or Tensorflow to make construction of the network as straightforward as possible.

\section{Technologies Explored}
Here we should explain some of the other technolgies out there for doing the same thing, that we have chosen not to use, why they are not good enough. (Digital card games circumvent the need for image recognition but involve a learning curve to use the interface. Not too sure at the moment what other technologies do this.)



\nocite{*}
\bibliographystyle{plain}
\bibliography{proposal}

\end{document}

% http://archive.ics.uci.edu/ml/datasets/Heart+Disease
