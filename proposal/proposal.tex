\documentclass{article}
\usepackage{amsmath,amssymb,amsbsy}
\usepackage{graphicx}
\usepackage[letterpaper,lmargin=2.5cm,rmargin=2.5cm,tmargin=2.5cm,bmargin=2.5cm]{geometry}
\usepackage{color}
\usepackage{listings}
\usepackage{listings-python-options}
\usepackage{subfig}

% some useful definitions:

\newcommand{\tr}{\intercal} % transpose
% we use boldface math characters often so let's create some shortcuts:
\newcommand{\X}{\mathbf{X}}
\newcommand{\x}{\mathbf{x}}
\newcommand{\y}{\mathbf{y}}
\newcommand{\w}{\mathbf{w}}
% some useful operators
\newcommand{\argmax}[1]{\underset{#1}{\operatorname{argmax}}}
\newcommand{\argmin}[1]{\underset{#1}{\operatorname{argmin}}}




\begin{document}

\title{ Playing Card Image Recognition\\[1ex] \footnotesize\mdseries CS 545, Fall 2016 Project Proposal }
\author{Gregory Poisson, Bradley Winters, Joshua O'Dell}
\maketitle

\noindent\hrulefill
\vspace{-5mm} %to remove some whitespace before "Contents"
% \tableofcontents
% \noindent\hrulefill

\section{Motivation}
We will train a neural network to recognize playing cards within an image.  The goal being to eventually utilize this knowledge to create a system which a human could use as an opponent in a variety of games involving playing cards, for practice or potentially for competition.

The abstract strategies behind various card games would be beyond the scope of this implementation, but this model could be used to bootstrap a system whose goal was to learn games such as poker by being able to observe games played by humans and identify the cards they play.

\section{Tasks}
The first step of the project will involve significant data collection. We need a large number of images of playing cards to be used as training and as testing data. Since we will need to train over so many of them, these images cannot be of very high resolution. Our rough procedure for data collection is as follows: 

1) Photograph each playing card from ten different angles and zoom levels. Ensure each image is labeled according to the card in the image, and that for each card, at least one labeled corner of the card is visible. This should produce 520 labeled images. 

2) Using some simple scripting and the Linux convert tool, effects and transformations will be applied to copies of each of the images in order to increase the size of the dataset to roughly 10,000 labeled images. This will be the data we use for training and for testing. 

3) A convolutional neural network \cite{conv} will take as input the pixel values for each image, and training will iterate until a model converges which can correctly classify a card in a new image. We will experiment with regularization parameters to prevent the network from learning overly specific features, as well as with the size of the hidden layers. The network will have a number of inputs equal to the number of pixels in an image (currently planning something around the 150x150 range), and will have 52 outputs, one for each possible card.

It will be interesting to see if the network is able to identify multiple cards in the same image, and to see if it can make reasonable guesses about cards which may be partially obscured (ie, only the number is visible, but not the suit, etc).

To implement the neural network, we can use a modern machine learning framework such as Theano or Tensorflow to make construction of the network as straightforward as possible, while allowing for the use of modern tools such as GPU processing \cite{gpu2}, \cite{gpu}.


\section{Other Technologies Explored}
\subsection{Edge Detection}
There are several edge detection algorithms that could possible be used to locate a playing card within the image.  Some clipping and transformations can then be done to the image to make it more closely match a well known image.  After this we can run the image thorugh either a neural network or other pattern recognizer to determine which card most closely matches the image.

This solution requires quite a bit of minipulation of the image.  In practice it might be more effective, however we wish to focus on a solution that utilizes a Neural network, and gives us a chance to utilize techniques that are more `machine learning' focused.

\subsection{Geone}
Another option is to use Geons, which are simple objects that our brains are adapt at looking for.  The idea here is that the image can be scanned for a set of object that are known, like an image of a heart or club.  The results of this scan are then used in analysis.  if 4 hearts are found then chances are the card is a 4 of hearts.

We have decided not to use this technique, as it involves a more brute force way of looking thorough the image.  In practice this may be more efficient, since we have a finite set of objects, however we wish to focus on a solution that `learns' and utilizes more 

\section{Time-line and responsibilities}
We will break the project into 3 phases listed below.  Each member of the group will be responsible for a piece of the project in each phase.  This is a rough estimate on work load and can possibly change as the project continues.

\subsection{Experiment: Complete Nov 7}
Greg: create train / test data

Josh: create Neural Network classes

Brad: Create test bed and examine initial results

\subsection{Test: Complete Nov 25}
Brad: create automated framework for validating classification

Greg: examine results

Josh: graphs and presentation assets

\subsection{Report: Complete Dec 8}
Josh: create report

Brad: Poster

Greg: Video / in class Presentation ()

We are still unclear if we are to do a Video or an in class presentation given 2/3 of the team are distance students

\subsection{Presentation: Complete Dec 12}

Last minute preparations including sending poster to Greg, and finalizing the speech / presentation to be given in class.

\nocite{*}
\bibliographystyle{plain}
\bibliography{proposal}

\end{document}

% http://archive.ics.uci.edu/ml/datasets/Heart+Disease
